\documentclass[a4paper,12pt]{article}

\usepackage[utf8]{inputenc}
\usepackage[T1]{fontenc}
\usepackage[francais]{babel}

\usepackage{graphicx} % Inclusion d'images
\usepackage{amsfonts} % Symboles maths
\usepackage{amsmath} % Aligner les équations
\usepackage{fullpage} % Marges
\usepackage{xspace} % Espace après macros
\usepackage{float} % Positionnement des images

\title{Cadre du projet - Cannon Ball RC}
\author{COUTELOU Thibaut - MUGNIER Benjamin - PERRIN Guillaume}
\date{\today}

\begin{document}

\maketitle

%\tableofcontents

\section{Objectifs}

\subsection{Objectifs techniques}

L'objectif principal du projet est de rendre une voiture radio commandée
autonome (autour d'une tablette Windows 8). Le livrable comporte 3 objectifs principaux :
\begin{itemize}
    \item \emph{Sécurité}. Elle doit être garantie en permanence par une commande d'arrêt
        d'urgence (appui sur un bouton de la télécommande).
    \item \emph{Lièvre}. La voiture doit être capable de suivre un marqueur.
    \item \emph{Circuit}. La voiture doit être capable de parcourir un circuit délimité par des
        marqueurs.
\end{itemize}

\paragraph{}
Le livrable comporte également des objectifs secondaires :
\begin{itemize}
    \item \emph{Portage sur Android}. Rééquiper une voiture et utiliser une tablette
        Android pour la contrôler au lieu de d'une tablette Windows. Cela permet
        également de comparer leurs performances.
    \item \emph{Collectes de métriques}. La voiture doit être en mesure d'envoyer à tout instant
        des informations sur sa position, sa direction, sa propulsion, et d'autres
        résultats de capteurs jugés pertinents.
    \item \emph{Prise en compte des autres voitures}. La voiture doit être
        capable d'éviter ou de suivre d'autres voitures signalées
\end{itemize}

\paragraph{}
Ce projet étant très lié à la recherche, le but est de démontré qu'il est possible de
traiter des images suffisemment vite pour pouvoir guider une voiture en temps réel. De
plus, les voitures autonomes sont un pôle de recherche actuel très actif, ce projet
permet donc de répondre à certaines problématiques.

\subsection{Objectifs de délai}

Le projet doit être terminé le lundi 16 juin. La soutenance doit être préparée pour la
semaine du 16 juin.  Une pré-soutenance sera réalisé le vendredi 13 juin car Didier
Donsez n'est pas disponible la semaine suivante.

\subsection{Objectif expérimental}

Ce projet est financé par Intel dans le but de mettre au point des "kits
pédagogiques" qu'Intel pourrai fournir à d'autres universités.

\section{Méthodologie}

La méthode de développement choisie est proche de l'agilité : Des prototypes de
chaque fonction sont développés très rapidement, puis intégrés au système, et
enfin réécrits de manière plus propre. De nombreuses tâches sont critiques
dans ce projet comme tout projet FabLab.

En effet, une tâche qui parait simple comme faire avancer la voiture est en
réalité liée à de nombreuses autres tâches, à savoir le cablage et
l'implémentation du microcontrolleur ainsi que l'implémentation du signal
d'urgence.

\section{Ressources}

Le matériel à disposition est :
\begin{itemize}
    \item Deux voitures radiocommandées
    \item Une tablette Android
    \item Une tablette Windows
    \item Deux Arduinos
    \item Deux webcams
\end{itemize}

Les tablettes sont embarquées sur les voitures radio commandées. Elles font
tourner un logiciel qui acquière des images par la webcam, et calcul les ordres
à transmettre aux Arduinos. Les Arduinos contrôlent la vitesse de rotation du
moteur et la position des roues.

Nous disposons également du matériel du Fablab du bâtiment C, comme la découpeuse laser
ou l'imprimante 3D.

\section{Planning}

Le projet débute par une phase de rétro ingénierie sur l'existant ainsi que la prise en
main du travail déjà réalisé.

Le projet est ensuite découpé en 2 phases qui sont exécutées en parallèle :
\begin{itemize}
    \item Une phase de développement du logiciel pure : comprenant la reconnaissance de
        marqueurs, l'envoie de données, et le micro contrôleur
    \item Une phase de développement et de perfectionnement de l'intelligence
        artificielle utilisée ici pour suivre les différents marqueurs
\end{itemize}

On notera que la phase de conception est réduite ici au minimum du fait de notre mode de
fonctionnement agile mais aussi que c'est un projet Fablab et est donc très axé
prototypage.

La documentation est quand à elle rédigée tout au long du projet, exepté le cahier des
charges rédigé au tout début. 


\section{Organisation du projet}

L'équipe projet est constituée de Didier Donsez, le référent du projet, qui constitue
également le comité de pilotage client.
Le projet est réalisé en association avec Intel et son ingénieur Paul Guermonprez, qui
fournit du matériel servant à la réalisation celui-ci. Le FabLab du batiment C est tenu
par Jerôme Maisonnasse qui nous fournit également une expertise technique sur les
technologies embarquées. Ils constituent l'équipe étendue.

Jules Legros et Benoit Perruche sont deux étudiants de Polytech Grenoble qui
ont travaillé sur ce projet à l'itération précédente. Nous pouvons nous appuyer sur eux
en cas de besoin pour le développement.

\subsection{Communication interne}

Une réunion est organisée entre les trois membres de l'équipes tous les matins,
et en cas besoin. Elles sont nécessaires pour mettre en
commun ce qui a été réalisé et planifier les journées à venir.

\subsection{Rapports journaliers}

Tous les jours, le journal de bord est mis à jour et un mail est envoyé à
Didier Donsez pour le tenir au courant de l'avancement et des problèmes
rencontrés. Une réponse est en général apportée suite à ce mail et nous est très utile
pour le développement de la journée suivante.

\subsection{Contributions}

Le projet se veut open source sous licence MIT (c'est-à-dire que tout ce qui constitue le projet
doit être accessible et disponible à tous). Le code source est donc disponible
sur la plateforme github et se doit d'être commenté et clair. De même, une
documentation élaborée est de rigueur.

\section{Risques}

Le projet requiert du matériel pour être réalisé, nous sommes donc dépendant de l'arrivée
de celui pour travailler. Ainsi, nous ne pouvons pas, par exemple, faire des tests en
plein air sans avoir les plots fabriqués par le Fablab.

\subsection{Portage sur Android}

Le portage de notre solution sur Android peut poser des problèmes à différents
niveaux :
\begin{itemize}
    \item Communication avec l'Arduino.
    \item Utilisation des bibliothèques (Opencv, arUco)
    \item Portage du code à proprement parlé.
\end{itemize}

On peut limiter ce risque en démarrant le portage vers Android le plus tôt
possible et en utilisant des librairies multi plateformes.

Il faut également faire de la recherche (voire même un état de l'art) afin de s'assurer
que tout ce que nous utilisons à un instant t pourra bien être porté.

\subsection{Performances trop faibles}

Il se peut que les lourds calculs réalisés par la tablette rendent impossible
le traitement en temps réel.

Pour limiter ce risque, on peut utiliser des bibliothèques légères. Mais cela
risque de ne pas suffire.
On peut aussi utiliser des tablettes plus puissantes, si Intel nous les fourni.
Enfin, on peut réduire la taille de l'image traitée, mais la détection des
marqueurs risque d'être moins bonne.

\subsection{Aboutissement}

Le projet est à caractère de recherche. Il est voué à être reprit, comme nous l'avons
fait, par d'autres étudiants afin de le poursuivre. Nous ne sommes donc pas sûr d'avoir
un résultat final et risquons de bloquer sur un point. Les solutions ne sont pas connues
d'avance.

\section{Indicateurs}

Les indicateurs d'avancement des livrables sont tout d'abord que les fonctionnalités
attendues sont développées dans un délai raisonnable.

\paragraph{}
En ce qui concerne le respect de la démarche, on pourrait envisager un indicateur de
curiosité. En effet le projet n'étant pas sûr d'aboutir, il faut tester de nombreuses
pistes quitte à ne pas les suivre. Cette démarche est importante pour que dans l'avenir
les mêmes erreurs ne soient pas réitérées.
Ceci est nécessaire pour trouver des solutions. Plus nous testons, plus nous respectons
la démarche du projet.

\paragraph{}
La qualité des livrables peut être mesurée par la vitesse moyenne de la voiture sur un
circuit de référence. Plus la voiture le parcours rapidement et sans erreur, plus le
logiciel est de qualité.

\paragraph{}
La motivation de l'équipe est importante dans un projet, nous nous efforçons donc à venir
à 9 heures tous les matins. Cela permet aussi d'être sur que tout le monde est présent à
la réunion et que le matériel est bien ramené pour permettre à l'équipe de travailler.

\end{document}
