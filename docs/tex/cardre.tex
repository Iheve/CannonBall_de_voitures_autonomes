\documentclass[a4paper,12pt]{article}

\usepackage[utf8]{inputenc}
\usepackage[T1]{fontenc}
\usepackage[francais]{babel}

\usepackage{graphicx} % Inclusion d'images
\usepackage{amsfonts} % Symboles maths
\usepackage{amsmath} % Aligner les équations
\usepackage{fullpage} % Marges
\usepackage{xspace} % Espace après macros
\usepackage{float} % Positionnement des images

\title{Cadre du projet}
\author{COUTELOU Thibaut - MUGNIER Benjamin - PERRIN Guillaume}
\date{\today}

\begin{document}

\maketitle

%\tableofcontents

\section{Objectifs}

\subsection{Objectifs techniques}

L'objectif principal du projet est de rendre une voiture radiocommandée
autonome (autour d'une tablette Windows 8). Les livrables attendus sont :
\begin{itemize}
    \item Sécurité
    \item Lièvre
    \item Circuit
    \item Prise en compte des autres voitures
    \item Collectes de métriques
\end{itemize}

Un objectif secondaire est de monter une seconde voiture autour d'un  système
basé sur Android.
Cela permettra de comparer les performances entre ces deux systèmes.

\subsubsection{Sécurité}

La sécurité doit être garantie en permanence par une commande d'arrêt
d'urgence. (Appuyer sur un bouton de la télécommande).

\subsubsection{Lièvre}

La voiture doit être capable de suivre un marqueur.

\subsubsection{Circuit}

La voiture doit être capable de suivre un circuit délimité par des marqueurs.

\subsection{Objectifs de délai}

Le projet doit être terminé le lundi 16 juin.
La soutenance doit être préparée pour la semaine du 16 juin.
Une pré-soutenance doit être préparée pour le vendredi 13 juin car Didier
Donsez n'est pas disponible la semaine suivante

\subsection{Objectif expérimental}

Ce projet est financé par Intel dans le but de mettre au point des "kits
pédagogiques" qu'Intel pourrai fournir à d'autres universités.











\section{Méthodologie}
TODO









\section{Ressources}

\subsection{Matériel}

Le matériel à disposition est :
\begin{itemize}
    \item Deux voitures radiocommandées
    \item Une tablette Android
    \item Une tablette Windows
    \item Deux arduino
    \item Deux webcams
\end{itemize}

Les tablettes sont embarquées sur les voitures raio commandées. Elles font
tourner un logiciel qui acquière des images par la webcam, et calcul les ordres
à transmettre aux Arduinos. Les arduinos contrôlent la vitesse de rotation du
moteur et la position des roues.

Ce matériel est mis a disposition par Intel.

\subsection{Ressources humaines}

Trois étudiants en deuxième année à l'Ensimag travaillent à plein temps sur ce
projet pendant 3 semaines.










\section{Organisation du projet}

Didier Donsez est le maitre d'ouvrage du projet.
Le projet est réalisé en association avec Intel et son ingénieur Paul Guermonprez, qui
fournit du matériel servant à la réalisation celui-ci.

Jules Legros et Benoit Perruche sont deux étudiants de polytech Grenoble qui
ont travaillé sur ce projet avant nous.







\section{Communication}

\subsection{Interne}

Une réunion est organisée entre les trois membres de l'équipes tous le matins,
et à chaque fois que cela est nécessaire. Elles sont nécessaires pour mettre en
commun ce qui a été réalisé et planifier les journées à venir.

\subsection{Rapports journaliers}

Tous les jours, le journal de bord est mis à jour et un mail est envoyé à
Didier Donsez pour le tenir au courant de l'avancement et des problèmes
rencontrés.





\section{Risques}

\subsection{Portage sur Android}

Le portage de notre solution sur Android peut poser des problèmes à différents
niveaux :
\begin{itemize}
    \item Communication avec l'Arduino.
    \item Utilisation des bibliothèques (Opencv, aruco)
    \item Portage du code à proprement parlé.
\end{itemize}

On peut limiter ce risque en démarrant le portage vers Android le plus tôt
possible et en utilisant des librairies cross-platform.

\subsection{Performances trop faibles}

Il se peut que les lourds calculs réalisés par la tablette rendent impossible
le traitement en temps réels.

Pour limiter ce risque, on peut utiliser des bibliothèques légères. Mais cela
risque de ne pas suffire.
On peut aussi utiliser des tablettes plus puissantes, si Intel nous les fourni.


\section{Indicateurs}

Les indicateurs de réussite du projet sont :
\begin{itemize}
    \item La voiture réagit correctement aux marqueurs
    \item Le circuit est correctement parcouru
    \item La voiture évite les autres voitures
    \item Les métriques sont collectées
    \item Le traitement par la tablette est fluide
    \item L'équipe est présente à 9h tous les matins (motivation)
\end{itemize}




\end{document}
